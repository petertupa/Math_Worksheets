\documentclass{article}%
\usepackage{sagetex}
\usepackage{multicol}
\usepackage{tikz}
\usepackage{amsmath}
\setlength {\textwidth} {7.5in} 
\setlength {\textheight}{10.0in}
\setlength{\oddsidemargin} {-0.5in}
\setlength {\evensidemargin} {-0.5in}
\setlength{\topmargin} {-0.0in} 
\setlength {\headheight} {0.0pt} 
\setlength {\headsep} {0in}
\setlength{\columnsep}{0.5cm}


\begin{document}


\begin{sagesilent}
import random
# Generate and set a random seed.
t=ZZ.random_element(999999)
set_random_seed(t)
# Generate lists of random integers to be used
a=[(-1)^(ZZ.random_element(1,3))*ZZ.random_element(1,10) for i in range(12)]
b=[(-1)^(ZZ.random_element(1,3))*ZZ.random_element(1,10) for i in range(len(a))]
c=[(-1)^(ZZ.random_element(1,3))*ZZ.random_element(1,8) for i in range(len(a))]
d=[(-1)^(ZZ.random_element(1,3))*ZZ.random_element(2,10) for i in range(len(a))]
f=[(-1)^(ZZ.random_element(1,3))*ZZ.random_element(1,9) for i in range(len(a))]
g=[(-1)^(ZZ.random_element(1,3))*ZZ.random_element(2,34) for i in range(len(a))]
h=[(-1)^(ZZ.random_element(1,3))*ZZ.random_element(1,12) for i in range(len(a))]
k=[(-1)^(ZZ.random_element(1,3))*ZZ.random_element(1,12) for i in range(len(a))]
m=[ZZ.random_element(2,10) for i in range(len(a))]
n=[ZZ.random_element(4,10) for i in range(len(a))]
q=[ZZ.random_element(2,10) for i in range(len(a))]
r=[(-1)^(ZZ.random_element(1,3))*ZZ.random_element(1,10) for i in range(len(a))]
II=Permutations(range(len(a))).random_element()


# Setup numerators and denominators for each question form and store these each in a list
#Problem Type1: (x+a)/(x+b) * ((x+b)(x+c))/((x+a)(x+c)) = (x+c)/(x+d)
numer1L=[(x-a[i])  for i in range(len(a))]
numer1R=[expand((x-b[i])*(x-c[i]))  for i in range(len(a))]
denom1L=[(x-b[i])  for i in range(len(a))]
denom1R=[expand((x-a[i])*(x-d[i]))  for i in range(len(a))]
answer1=[(x-c[i])/(x-d[i]) for i in range(len(a))]

#Problem Type2: ((x+a)(x+b))/((x+a)(x+c)) * ((x+c)(x+d))/((x+f)(x+b)) = (x+d)/(x+f)
numer2L=[expand((x-a[i])*(x-b[i]))  for i in range(len(a))] 
numer2R=[expand((x-c[i])*(x-d[i]))  for i in range(len(a))] 
denom2L=[expand((x-a[i])*(x-c[i]))  for i in range(len(a))]
denom2R=[expand((x-f[i])*(x-b[i]))  for i in range(len(a))]
answer2=[(x-d[i])/(x-f[i]) for i in range(len(a))]

#Problem Type3: (x+a)/b  \div (fx+fa)/d = d/bf
numer3L=[(x-f[i])  for i in range(len(a))]
numer3R=[expand((q[i]*x-q[i]*f[i]))  for i in range(len(a))]
denom3L=[m[i]  for i in range(len(a))]
denom3R=[n[i]  for i in range(len(a))]
answer3=[ n[i]/(m[i]*q[i]) for i in range(len(a))]

#Problem Type4:  (x+a)(x+b) \div ((x+a)(x-a))/(x+c) = (x+b)(x+c)/(x-a)
numer4L=[expand((x-f[i])*(x-c[i]))  for i in range(len(a))] 
numer4R=[expand((x-f[i])*(x+f[i]))  for i in range(len(a))] 
denom4R=[(x-b[i])  for i in range(len(a))]
answer4=[( ((x-f[i])*(x-c[i])*(x-b[i])) / ( (x-f[i])*(x+f[i]) )) for i in range(len(a))]

#Problem Type5: (x+a)/(x+b) \div ((x+a)(x+c))/((x+b)(x+d)) = (x+d)/(x+c)
numer5L=[expand((x-k[i]))  for i in range(len(a))] 
numer5R=[expand((x-k[i])*(x-f[i]))  for i in range(len(a))] 
denom5L=[(x-d[i])  for i in range(len(a))]
denom5R=[expand((x-d[i])*(x-c[i]))  for i in range(len(a))]
answer5=[( ((x-k[i])*(x-d[i])*(x-c[i])) / ( (x-k[i])*(x-f[i])*(x-d[i]) )) for i in range(len(a))]

#Problem Type6: ((x+a)(x+b))/((x+a)(x+c)) \div ((x+d)(x+b))/((x+f)(x+c)) =  (x+f)/(x+d)
numer6L=[expand((x-c[i])*(x-d[i]))  for i in range(len(a))] 
numer6R=[expand((x-k[i])*(x-d[i]))  for i in range(len(a))] 
denom6L=[expand((x-c[i])*(x-b[i]))  for i in range(len(a))]
denom6R=[expand((x-f[i])*(x-b[i]))  for i in range(len(a))]
answer6=[( ( (x-c[i])*(x-d[i])*(x-f[i])*(x-b[i]) ) / ( (x-c[i])*(x-b[i])*(x-k[i])*(x-d[i]))) for i in range(len(a))]


# assign variables to a null character
qoutput1=""
aoutput1=""
qoutput2=""
aoutput2=""
qoutput3=""
aoutput3=""
qoutput4=""
aoutput4=""
qoutput5=""
aoutput5=""
qoutput6=""
aoutput6=""



# generate the latex code for the questions and answers

for i in range(len(a)):
  qoutput1 += r" \item $%s \cdot %s$" %(latex(numer1L[i]/denom1L[i]),latex(numer1R[i]/denom1R[i])) 
  aoutput1 += r" \item $%s$" %(latex(answer1[i]))
  qoutput2 += r" \item $%s \cdot %s$" %(latex(numer2L[i]/denom2L[i]),latex(numer2R[i]/denom2R[i])) 
  aoutput2 += r" \item $%s$" %(latex(answer2[i]))
  qoutput3 += r" \item $\frac{%s}{%s} \div \frac{%s}{%s}$" %(latex(numer3L[i]),latex(denom3L[i]),latex(numer3R[i]),latex(denom3R[i])) 
  aoutput3 += r" \item $%s$" %(latex(answer3[i]))
  qoutput4 += r" \item $(%s) \div \frac{%s}{%s}$" %(latex(numer4L[i]),latex(numer4R[i]),latex(denom4R[i])) 
  aoutput4 += r" \item $%s$" %(latex(answer4[i])) 
  qoutput5 += r" \item $\frac{%s}{%s} \div \frac{%s}{%s}$" %(latex(numer5L[i]),latex(denom5L[i]),latex(numer5R[i]),latex(denom5R[i])) 
  aoutput5 += r" \item $%s$" %(latex(answer5[i])) 
  qoutput6 += r" \item $\frac{%s}{%s} \div \frac{%s}{%s}$" %(latex(numer6L[i]),latex(denom6L[i]),latex(numer6R[i]),latex(denom6R[i])) 
  aoutput6 += r" \item $%s$" %(latex(answer6[i])) 
  
  
\end{sagesilent}


%%%%%%%%%%%%%%%%%%%%  Latex Output %%%%%%%%%%%%%%%%%%%%

% Random Seed can be used to re-create a particular worksheet
\noindent Muliply or divide, as indicated, and then simplify if possible: \\
\textcolor{gray}{Random Seed: $\sage{t}$}


% Create list of problems via qoutput
\begin{multicols}{3}
\begin{enumerate}
\sagestr{qoutput1}
\sagestr{qoutput2}
\sagestr{qoutput3}
\sagestr{qoutput4}
\sagestr{qoutput5}
\sagestr{qoutput6}
\end{enumerate}
\end{multicols}


\vfill
\newpage


% Create associated list of solutions via aoutput
\begin{multicols}{3}
\begin{enumerate}
\sagestr{aoutput1}
\sagestr{aoutput2}
\sagestr{aoutput3}
\sagestr{aoutput4}
\sagestr{aoutput5}
\sagestr{aoutput6}
\end{enumerate}
\end{multicols}


\end{document}