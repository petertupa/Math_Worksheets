\documentclass{article}%
\usepackage{sagetex}
\usepackage{multicol}
\usepackage{tikz}
\usepackage{amsmath}
\setlength {\textwidth} {7.5in} 
\setlength {\textheight}{10.0in}
\setlength{\oddsidemargin} {-0.5in}
\setlength {\evensidemargin} {-0.5in}
\setlength{\topmargin} {-0.0in} 
\setlength {\headheight} {4.0pt} 
\setlength {\headsep} {0in}
\setlength{\columnsep}{0.5cm}


\begin{document}


\begin{sagesilent}
import random
# Generate and set a random seed.
t=ZZ.random_element(999999)
#t=int( open('random_seed.txt').read() ) 
set_random_seed(t)
# Generate lists of random integers to be used
a=[(-1)^(ZZ.random_element(1,3))*ZZ.random_element(1,20) for i in range(15)]
b=[(-1)^(ZZ.random_element(1,3))*ZZ.random_element(1,20) for i in range(len(a))]
c=[(-1)^(ZZ.random_element(1,3))*ZZ.random_element(1,10) for i in range(len(a))]
d=[(-1)^(ZZ.random_element(1,3))*ZZ.random_element(2,10) for i in range(len(a))]
f=[(-1)^(ZZ.random_element(1,3))*ZZ.random_element(1,10) for i in range(len(a))]
g=[(-1)^(ZZ.random_element(1,3))*ZZ.random_element(2,10) for i in range(len(a))]
h=[(-1)^(ZZ.random_element(1,3))*ZZ.random_element(1,12) for i in range(len(a))]
k=[(-1)^(ZZ.random_element(1,3))*ZZ.random_element(1,12) for i in range(len(a))]
m=[(-1)^(ZZ.random_element(1,3))*ZZ.random_element(1,12) for i in range(len(a))]
n=[(-1)^(ZZ.random_element(1,3))*ZZ.random_element(1,12) for i in range(len(a))]
p=[(-1)^(ZZ.random_element(1,3))*ZZ.random_element(1,12) for i in range(len(a))]
q=[(-1)^(ZZ.random_element(1,3))*ZZ.random_element(1,12) for i in range(len(a))]


# Setup list of inequality types and random choice of sign for each index
# Note Indices of i:  0 for <, 1 for <=, 2 for >, 3 for >=
LINALT=('<','<=','>','>=')
LIN=['<','\leq','>','\geq']

# R_andom IN_equality Q_uestion/A_nswer Type# A(left)/B(right)
RINQ1A=[ZZ.random_element(0,4) for i in range(len(a))] #Initial random inequality switch for left/A ineq
RINQ1B=[ZZ.random_element(0,4) for i in range(len(a))] #Initial random inequality switch for right/B ineq                
RINQ2A=[ZZ.random_element(0,4) for i in range(len(a))]
RINQ2B=[ZZ.random_element(0,4) for i in range(len(a))]
RINA2A=[RINQ2A[i] for i in range(len(a))]              #Inequality switch after accounting for negative x-coefficient, if needed
RINA2B=[RINQ2B[i] for i in range(len(a))]
RINQ3A=[ZZ.random_element(0,4) for i in range(len(a))]
RINQ3B=[ZZ.random_element(0,4) for i in range(len(a))]
RINQ4A=[ZZ.random_element(0,4) for i in range(len(a))]
RINQ4B=[ZZ.random_element(0,4) for i in range(len(a))]
RINA4A=[RINQ4A[i] for i in range(len(a))]
RINA4B=[RINQ4B[i] for i in range(len(a))]
RINQ5A=[ZZ.random_element(0,2) for i in range(len(a))]
RINQ5B=[ZZ.random_element(0,2) for i in range(len(a))]
RINQ6A=[ZZ.random_element(0,2) for i in range(len(a))]
RINQ6B=[ZZ.random_element(0,2) for i in range(len(a))]

# check for conditions that would result in
# incorrect inequality
for j in range(len(a)):
  while a[j]==b[j]:
    a[j]=(-1)^(ZZ.random_element(1,3))*ZZ.random_element(1,20)
    b[j]=(-1)^(ZZ.random_element(1,3))*ZZ.random_element(1,20)

for j in range(len(a)):
  while m[j]>=n[j]:
    m[j]=(-1)^(ZZ.random_element(1,3))*ZZ.random_element(1,12)
    n[j]=(-1)^(ZZ.random_element(1,3))*ZZ.random_element(1,12)

for j in range(len(a)):
  while p[j]>=q[j]:
    p[j]=(-1)^(ZZ.random_element(1,3))*ZZ.random_element(1,12)
    q[j]=(-1)^(ZZ.random_element(1,3))*ZZ.random_element(1,12)    

    
    
# Returns string formated solution for two AND inequalities  
# Assumes input values, A & B, are in solved forms
# A is associated with SL inequality sign 
# B is associated with SR inequality sign 
def INEQ_INTERSECTION(A,B,SL,SR):
  global INEQ_INT
  INEQ_INT=""
  if SL>=2 and SR<=1:
    if A>B: STR="\emptyset"
    elif SL==2 and SR==0:   STR="(" + str(A) + "," + str(B) + ")"
    elif SL==3 and SR==0:   STR="[" + str(A) + "," + str(B) + ")"
    elif SL==2 and SR==1:   STR="(" + str(A) + "," + str(B) + "]"
    elif SL==3 and SR==1:   STR="[" + str(A) + "," + str(B) + "]"
    else: STR="Err: INEQ_INTERSECTION: 1st IF branch"   
  elif SL<=1 and SR>=2:
    if A<B: STR="\emptyset"
    elif SL==0 and SR==2:   STR="(" + str(B) + "," + str(A) + ")" 
    elif SL==1 and SR==2:   STR="(" + str(B) + "," + str(A) + "]"
    elif SL==0 and SR==3:   STR="[" + str(B) + "," + str(A) + ")"
    elif SL==1 and SR==3:   STR="[" + str(B) + "," + str(A) + "]"    
    else: STR="Err: INEQ_INTERSECTION: 2nd IF branch" 
  elif SL<=1 and SR<=1:
    if A>B:
      if SR==0:   STR="(-\infty," + str(B) + ")"
      elif SR==1:   STR="(-\infty," + str(B) + "]"
    elif A<B:
      if SL==0:   STR="(-\infty," + str(A) + ")"
      elif SL==1:   STR="(-\infty," + str(A) + "]"      
    else: STR="Err: INEQ_INTERSECTION: 3rd IF branch" 
  elif SL>=2 and SR>=2:
    if A>B:
      if SL==2:     STR="(" + str(A) + ",\infty)"
      elif SL==3:   STR="[" + str(A) + ",\infty)"
    elif A<B:
      if SR==2:     STR="(" + str(B) + ",\infty)"
      elif SR==3:   STR="[" + str(B) + ",\infty)"
    else: STR="Err: INEQ_INTERSECTION: 4th IF branch"
  else:
    STR="Err: INEQ_INTERSECTION: Missed all IFs"
    
  INEQ_INT=STR
  print(INEQ_INT)
  return INEQ_INT

 
# Returns string formated solution for two OR inequalities  
# Assumes input values, A & B, are in solved forms
# A is associated with SL inequality sign 
# B is associated with SR inequality sign  
def INEQ_UNION(A,B,SL,SR):
  global INEQ_UN
  INEQ_UN=""
  if SL>=2 and SR<=1:
    if A<B: STR="(-\infty,\infty)"
    elif SL==2 and SR==0:   STR="(-\infty," + str(B) + ") \cup (" + str(A) + ",\infty)"
    elif SL==3 and SR==0:   STR="(-\infty," + str(B) + ") \cup [" + str(A) + ",\infty)"
    elif SL==2 and SR==1:   STR="(-\infty," + str(B) + "] \cup (" + str(A) + ",\infty)"
    elif SL==3 and SR==1:   STR="(-\infty," + str(B) + "] \cup [" + str(A) + ",\infty)"
    else: STR="Err: INEQ_UNION: 1st IF branch"   
  elif SL<=1 and SR>=2:
    if A>B: STR="(-\infty,\infty)"
    elif SL==0 and SR==2:   STR="(-\infty," + str(A) + ") \cup (" + str(B) + ",\infty)" 
    elif SL==1 and SR==2:   STR="(-\infty," + str(A) + "] \cup (" + str(B) + ",\infty)"
    elif SL==0 and SR==3:   STR="(-\infty," + str(A) + ") \cup [" + str(B) + ",\infty)"
    elif SL==1 and SR==3:   STR="(-\infty," + str(A) + "] \cup [" + str(B) + ",\infty)"    
    else: STR="Err: INEQ_UNION: 2nd IF branch"   
  elif SL<=1 and SR<=1:
    if A>B:
      if SL==0:   STR="(-\infty," + str(A) + ")"
      elif SL==1:   STR="(-\infty," + str(A) + "]"
    elif A<B:
      if SR==0:   STR="(-\infty," + str(B) + ")"
      elif SR==1:   STR="(-\infty," + str(B) + "]"      
    else: STR="Err: INEQ_UNION: 3rd IF branch"  
  elif SL>=2 and SR>=2:
    if A>B:
      if SR==2:     STR="(" + str(B) + ",\infty)"
      elif SR==3:   STR="[" + str(B) + ",\infty)"
    elif A<B:
      if SL==2:     STR="(" + str(A) + ",\infty)"
      elif SL==3:   STR="[" + str(A) + ",\infty)"
    else: STR="Err: INEQ_UNION: 4th IF branch"
  else:
    STR="Err: INEQ_UNION: Missed all IFs"
    
  INEQ_UN=STR
  print(INEQ_UN)
  return INEQ_UN 

# Returns string formated solution for 3 part compound inequality  
# Assumes input values, A & B, are in solved forms
# A is associated with SL inequality sign 
# B is associated with SR inequality sign
# If PM (plus/minus) is negative, function reverses inequality
def INEQ_TRIPLE(A,B,SL,SR,PM): 
  global INEQ_TRPL
  INEQ_TRPL="" 
  if PM>0:
    if   SL==0 and SR==0:    STR="(" + str(A) + "," + str(B) + ")"
    elif SL==0 and SR==1:    STR="(" + str(A) + "," + str(B) + "]"
    elif SL==1 and SR==0:    STR="[" + str(A) + "," + str(B) + ")"
    elif SL==1 and SR==1:    STR="[" + str(A) + "," + str(B) + "]"
  elif PM<0:
    if   SR==0 and SL==0:    STR="(" + str(B) + "," + str(A) + ")"
    elif SR==0 and SL==1:    STR="(" + str(B) + "," + str(A) + "]"
    elif SR==1 and SL==0:    STR="[" + str(B) + "," + str(A) + ")"
    elif SR==1 and SL==1:    STR="[" + str(B) + "," + str(A) + "]"
  
  INEQ_TRPL=STR
  print(INEQ_TRPL)
  return INEQ_TRPL
  

# Returns opposite sign inequality of same type
# Use for setting up paired inequality
# or when mulitplication by a negative occurs
def INEQ_SWTCH(A):
  global INEQPR
  if   A == 0:    INEQSW = 2
  elif A == 1:    INEQSW = 3
  elif A == 2:    INEQSW = 0
  elif A == 3:    INEQSW = 1
  
  return INEQSW  
  
  
# If the coef of x is negative, we need a sign switch  
for i in range(len(a)):  
  if c[i] < 0:
    RINA2A[i]=INEQ_SWTCH( RINQ2A[i])
  
  if f[i] < 0:
    RINA2B[i]=INEQ_SWTCH( RINQ2B[i])
  
  if k[i] < 0:
    RINA4A[i]=INEQ_SWTCH( RINQ4A[i])
  
  if d[i] < 0:
    RINA4B[i]=INEQ_SWTCH( RINQ4B[i])
  
  
  
  
#for i in range(len(a)):  
#  if c[i] < 0:  
    #if RINQ2A[i] == 0:      RINA2A[i] = 2
    #if RINQ2A[i] == 1:      RINA2A[i] = 3
    #if RINQ2A[i] == 2:      RINA2A[i] = 0
    #if RINQ2A[i] == 3:      RINA2A[i] = 1  
    
#for i in range(len(a)):  
#  if f[i] < 0:
#    if RINQ2B[i] == 0:      RINA2B[i] = 2
#    if RINQ2B[i] == 1:      RINA2B[i] = 3
#    if RINQ2B[i] == 2:      RINA2B[i] = 0
#    if RINQ2B[i] == 3:      RINA2B[i] = 1  
    
#for i in range(len(a)):  
#  if k[i] < 0:
#    if RINQ4A[i] == 0:      RINA4A[i] = 2
#    if RINQ4A[i] == 1:      RINA4A[i] = 3
#    if RINQ4A[i] == 2:      RINA4A[i] = 0
#    if RINQ4A[i] == 3:      RINA4A[i] = 1 
    
#for i in range(len(a)):  
#  if d[i] < 0:
#    if RINQ4B[i] == 0:      RINA4B[i] = 2
#    if RINQ4B[i] == 1:      RINA4B[i] = 3
#    if RINQ4B[i] == 2:      RINA4B[i] = 0
#    if RINQ4B[i] == 3:      RINA4B[i] = 1     
  
answer1=[INEQ_INTERSECTION(a[i],b[i],RINQ1A[i],RINQ1B[i]) for i in range(len(a))] 
answer2=[INEQ_INTERSECTION(  (a[i]-d[i])/c[i] , (b[i]-g[i])/f[i],RINA2A[i],RINA2B[i]) for i in range(len(a))] 
answer3=[INEQ_UNION(b[i],f[i],RINQ3A[i],RINQ3B[i]) for i in range(len(a))] 
answer4=[INEQ_UNION( (b[i]-a[i])/k[i],(h[i]-c[i])/d[i],RINA4A[i],RINA4B[i]) for i in range(len(a))] 
answer5=[INEQ_TRIPLE(m[i],n[i],RINQ5A[i],RINQ5B[i],1) for i in range(len(a))] 
answer6=[INEQ_TRIPLE( (p[i]-k[i])/h[i] , (q[i]-k[i])/h[i]   ,RINQ6A[i],RINQ6B[i], h[i]) for i in range(len(a))] 



# assign these two variables to a null character
qoutput1=""
aoutput1=""
qoutput2=""
aoutput2=""
qoutput3=""
aoutput3=""
qoutput4=""
aoutput4=""
qoutput5=""
aoutput5=""
qoutput6=""
aoutput6=""

# generate the latex code for the questions qoutput
# and the answers aoutput
for i in range(len(a)):
  qoutput1 += r" \item $x %s %s$ and $x %s %s$" %(LIN[RINQ1A[i]],a[i],LIN[RINQ1B[i]],b[i] )
  qoutput2 += r" \item $%s %s %s$ and $ %s %s %s$" %(latex(c[i]*x+d[i]),LIN[RINQ2A[i]],a[i],latex(f[i]*x+g[i]),LIN[RINQ2B[i]],b[i] )
  qoutput3 += r" \item $x %s %s$ or $x %s %s$" %(LIN[RINQ3A[i]],b[i],LIN[RINQ3B[i]],f[i] )
  qoutput4 += r" \item $%s %s %s$ or $ %s %s %s$" %(latex(k[i]*x+a[i]),LIN[RINQ4A[i]],b[i],latex(d[i]*x+c[i]),LIN[RINQ4B[i]],h[i] )
  qoutput5 += r" \item $ %s %s x %s %s$" %(m[i],LIN[RINQ5A[i]],LIN[RINQ5B[i]],n[i] )
  qoutput6 += r" \item $ %s %s %s %s %s$" %(p[i],LIN[RINQ6A[i]], latex(h[i]*x+k[i])   ,LIN[RINQ6B[i]],q[i] )
  aoutput1 += r" \item $%s$" %(answer1[i])
  aoutput2 += r" \item $%s$" %(answer2[i])
  aoutput3 += r" \item $%s$" %(answer3[i])
  aoutput4 += r" \item $%s$" %(answer4[i])
  aoutput5 += r" \item $%s$" %(answer5[i])
  aoutput6 += r" \item $%s$" %(answer6[i])

  
\end{sagesilent}






%%%%%%%%%%%%%%%%%%%%  Latex Output %%%%%%%%%%%%%%%%%%%%

% Random Seed can be used to re-create a particular worksheet
\noindent Solve each compound inequality. Except for the empty set, express the solution set in interval notation: \\
\textcolor{gray}{Random Seed: $\sage{t}$}


% Create list of problems via qoutput
\begin{multicols}{3}
\begin{enumerate}
\sagestr{qoutput1}
\sagestr{qoutput2}
\sagestr{qoutput3}
\sagestr{qoutput4}
\sagestr{qoutput5}
\sagestr{qoutput6}
\end{enumerate}
\end{multicols}


\newpage


% Create associated list of solutions via aoutput
\begin{multicols}{3}
\begin{enumerate}
\sagestr{aoutput1}
\sagestr{aoutput2}
\sagestr{aoutput3}
\sagestr{aoutput4}
\sagestr{aoutput5}
\sagestr{aoutput6}
\end{enumerate}
\end{multicols}


\end{document}