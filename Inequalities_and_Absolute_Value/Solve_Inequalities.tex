\documentclass{article}%
\usepackage{sagetex}
\usepackage{multicol}
\usepackage{tikz}
\usepackage{amsmath}
\setlength {\textwidth} {7.5in} 
\setlength {\textheight}{10.0in}
\setlength{\oddsidemargin} {-0.5in}
\setlength {\evensidemargin} {-0.5in}
\setlength{\topmargin} {-0.0in} 
\setlength {\headheight} {4.0pt} 
\setlength {\headsep} {0in}
\setlength{\columnsep}{0.5cm}


\begin{document}


\begin{sagesilent}
import random
# Generate and set a random seed.
t=ZZ.random_element(999999)
set_random_seed(t)
# Generate lists of random integers to be used
a=[(-1)^(ZZ.random_element(1,3))*ZZ.random_element(1,10) for i in range(20)]
b=[(-1)^(ZZ.random_element(1,3))*ZZ.random_element(1,15) for i in range(len(a))]
c=[(-1)^(ZZ.random_element(1,3))*ZZ.random_element(1,8) for i in range(len(a))]
d=[(-1)^(ZZ.random_element(1,3))*ZZ.random_element(2,20) for i in range(len(a))]
f=[(-1)^(ZZ.random_element(1,3))*ZZ.random_element(1,8) for i in range(len(a))]
g=[(-1)^(ZZ.random_element(1,3))*ZZ.random_element(2,34) for i in range(len(a))]
h=[(-1)^(ZZ.random_element(1,3))*ZZ.random_element(1,12) for i in range(len(a))]
k=[(-1)^(ZZ.random_element(1,3))*ZZ.random_element(1,12) for i in range(len(a))]
m=[random_prime(10,2) for i in range(len(a))]
n=[random_prime(10,2) for i in range(len(a))]
p=[ZZ.random_element(1,4) for i in range(len(a))]
q=[ZZ.random_element(1,10) for i in range(len(a))]
r=[(-1)^(ZZ.random_element(1,3))*ZZ.random_element(1,10) for i in range(len(a))]


# Setup list of inequality types and random choice of sign for each index
LIN=['>','\geq','<','\leq']
RIN1=[ZZ.random_element(0,4) for i in range(len(a))]
RIN2=[ZZ.random_element(0,4) for i in range(len(a))]
RIN3=[ZZ.random_element(0,4) for i in range(len(a))]

# check for conditions that would result in
# an infinite number of solutions and regenerate
# the coefficients until an equation with a solution
# is generated
for j in range(len(a)):
  while ((a[j]+c[j])-(f[j]+h[j]))==0:
    a[j]=(-1)^(ZZ.random_element(1,3))*(ZZ.random_element(1,6))
    c[j]=(-1)^(ZZ.random_element(1,3))*(ZZ.random_element(1,6))
    f[j]=(-1)^(ZZ.random_element(1,3))*(ZZ.random_element(1,6))
    h[j]=(-1)^(ZZ.random_element(1,3))*(ZZ.random_element(1,6))



# assign the left and right expressions to variables and store these each in a list
# Problem Type1: ax+b > c
expression1L=[c[i]*x+h[i] for i in range(len(a))]

# Problem Type2: ax+b > cx+d
expression2L=[(a[i]*x+b[i]+c[i]*x+d[i]) for i in range(len(a))]
expression2R=[(f[i]*x+g[i]+h[i]*x+k[i]) for i in range(len(a))]

# Problem Type3: ax/b + c/d > (fx+g)/k     
#Factors in denominators choosen for nicer LCD
# NUM and DEN setup for LateX output
expression3L=[(a[i]*x)/(n[i]*p[i])+b[i]/(m[i]*n[i]) for i in range(len(a))]
expression3R=[(k[i]*x+f[i])/(m[i]*n[i]*p[i])  for i in range(len(a))]
expression3RNUM=[(k[i]*x+f[i])  for i in range(len(a))]
expression3RDEN=[(m[i]*n[i]*p[i])  for i in range(len(a))]


# generate the solution of each problem via sagemath solve
# If statements used to select randomly choosen inequality sign
answer1=[x for i in range(len(a))]   
for i in range(len(a)):
  if RIN1[i]==0:
    answer1[i]=solve(expression1L[i]>g[i],x)[0]
  elif RIN1[i]==1:
    answer1[i]=solve(expression1L[i]>=g[i],x)[0]
  elif RIN1[i]==2:
    answer1[i]=solve(expression1L[i]<g[i],x)[0]
  elif RIN1[i]==3:
    answer1[i]=solve(expression1L[i]<=g[i],x)[0]
    

answer2=[x for i in range(len(a))]   
for i in range(len(a)):
  if RIN2[i]==0:
    answer2[i]=solve(expression2L[i]>expression2R[i],x)[0]
  elif RIN2[i]==1:
    answer2[i]=solve(expression2L[i]>=expression2R[i],x)[0]
  elif RIN2[i]==2:
    answer2[i]=solve(expression2L[i]<expression2R[i],x)[0]
  elif RIN2[i]==3:
    answer2[i]=solve(expression2L[i]<=expression2R[i],x)[0] 


answer3=[x for i in range(len(a))]   
for i in range(len(a)):
  if RIN3[i]==0:
    answer3[i]=solve(expression3L[i]>expression3R[i],x)[0]
  elif RIN3[i]==1:
    answer3[i]=solve(expression3L[i]>=expression3R[i],x)[0]
  elif RIN3[i]==2:
    answer3[i]=solve(expression3L[i]<expression3R[i],x)[0]
  elif RIN3[i]==3:
    answer3[i]=solve(expression3L[i]<=expression3R[i],x)[0]
    
    
    
# assign variables to a null character
qoutput1=""
aoutput1=""
qoutput2=""
aoutput2=""
qoutput3=""
aoutput3=""


# generate the latex code for the questions and answers
for i in range(len(a)):
  qoutput1 += r" \item $%s %s %s$" %(latex(expression1L[i]),LIN[RIN1[i]],latex(g[i]))
  aoutput1 += r" \item $%s$" %(latex(answer1[i][0]))
  
  qoutput2 += r" \item $%s %s %s$" %(latex(expression2L[i]),LIN[RIN2[i]],latex(expression2R[i]))
  aoutput2 += r" \item $%s$" %(latex(answer2[i][0]))

  qoutput3 += r" \item $%s %s \frac{%s}{%s} $" %(latex(expression3L[i]),LIN[RIN3[i]],latex(expression3RNUM[i]),latex(expression3RDEN[i])  )
  aoutput3 += r" \item $%s$" %(latex(answer3[i][0]))



\end{sagesilent}






%%%%%%%%%%%%%%%%%%%%  Latex Output %%%%%%%%%%%%%%%%%%%%

% Random Seed can be used to re-create a particular worksheet
\noindent Solve the given inequality: \\
\textcolor{gray}{Random Seed: $\sage{t}$}


% Create list of problems via qoutput
\begin{multicols}{3}
\begin{enumerate}
\sagestr{qoutput1}
\sagestr{qoutput2}
\sagestr{qoutput3}
\end{enumerate}
\end{multicols}


\vfill
\newpage


% Create associated list of solutions via aoutput
\begin{multicols}{3}
\begin{enumerate}
\sagestr{aoutput1}
\sagestr{aoutput2}
\sagestr{aoutput3}
\end{enumerate}
\end{multicols}


\end{document}