\documentclass{article}%
\usepackage{sagetex}
\usepackage{multicol}
\usepackage{tikz}
\usepackage{amsmath}
\setlength {\textwidth} {7.5in} 
\setlength {\textheight}{10.0in}
\setlength{\oddsidemargin} {-0.5in}
\setlength {\evensidemargin} {-0.5in}
\setlength{\topmargin} {-0.0in} 
\setlength {\headheight} {4.0pt} 
\setlength {\headsep} {0in}
\setlength{\columnsep}{0.5cm}


\begin{document}


\begin{sagesilent}
import random
# Generate and set a random seed.
t=ZZ.random_element(999999)
set_random_seed(t)
# Generate lists of random integers to be used
a=[(-1)^(ZZ.random_element(1,3))*ZZ.random_element(1,20) for i in range(20)]
b=[(-1)^(ZZ.random_element(1,3))*ZZ.random_element(1,20) for i in range(len(a))]
c=[(-1)^(ZZ.random_element(1,3))*ZZ.random_element(1,32) for i in range(len(a))]
d=[(-1)^(ZZ.random_element(1,3))*ZZ.random_element(2,20) for i in range(len(a))]
f=[(-1)^(ZZ.random_element(1,3))*ZZ.random_element(1,8) for i in range(len(a))]
g=[(-1)^(ZZ.random_element(1,3))*ZZ.random_element(2,34) for i in range(len(a))]
h=[(-1)^(ZZ.random_element(1,3))*ZZ.random_element(1,12) for i in range(len(a))]
k=[(-1)^(ZZ.random_element(1,3))*ZZ.random_element(1,12) for i in range(len(a))]
# Generate list of 1s and 2s for functions to pick >/< or >=/<=
INQL=[ZZ.random_element(1,3) for i in range(len(a))]
INQR=[ZZ.random_element(1,3) for i in range(len(a))]

# Ensure that left number in compound inequality is less than right number
for j in range(len(a)):
  while a[j]>=b[j]:
    a[j]=(-1)^(ZZ.random_element(1,3))*ZZ.random_element(1,20)
    b[j]=(-1)^(ZZ.random_element(1,3))*ZZ.random_element(1,20)



# assign variables to a null character
qoutput1=""
aoutput1=""
qoutput2=""
aoutput2=""
qoutput3=""
aoutput3=""


# This function returns a string format of compound 
# inequality of the form:  a<x<b
#SL is for left inequality
#SR is for right inequality
def INEQ_AB(A,B,SL,SR):
 global INEQ_SB
 INEQ_SB=""
 STL=""
 STR=""
 if (SL == 1):
   STL="<"
 else:  
   STL=" \leq "
      
 if (SR == 1):
   STR="<"
 else: 
   STR=" \leq "

 INEQ_SB=str(A) + STL + " x " + STR + str(B)
 print(INEQ_SB)
 return INEQ_SB

       
#This function returns a string format for
# interval notation between two endpoints 
def INEQ_INT(A,B,SL,SR):
 global INEQ_IN
 INEQ_IN=""
 STL=""
 STR=""
 if (SL == 1):
   STL="("
 else:
   STL="["
       
 if (SR == 1):
   STR=")"
 else: 
   STR="]"
 
 INEQ_IN= STL + str(A)+"," + str(B) + STR
 print(INEQ_IN)
 return INEQ_IN       
       
       
#This function returns a string format for
# an inequality of the form: x>c
def INEQ_PINF(C,SL):
  global INEQ_PI
  INEQ_PI=""
  STL=""
  if (SL == 1):
    STL=">"
  else:
    STL="\geq" 
    
  INEQ_PI="x" + STL + str(C)
  print(INEQ_PI)
  return  INEQ_PI
  
  
#This function returns a string format for
# an inequality of the form: x<d  
def INEQ_NINF(D,SR):
  global INEQ_NI
  INEQ_NI=""
  STR=""
  if (SR == 1):
    STR="<"
  else:
    STR="\leq" 
    
  INEQ_NI="x" + STR + str(D)
  print(INEQ_NI)
  return  INEQ_NI  
       
#This function returns a string format for
# interval notation of the form: (c,infty)        
def INEQ_PINF_INT(C,SL):
  global INEQ_PI_INT
  INEQ_PI_INT=""
  STL=""
  if (SL == 1):
    STL="("
  else:
    STL="[" 
    
  INEQ_PI_INT= STL + str(C) 
  print(INEQ_PI_INT)
  return  INEQ_PI_INT
  
  
#This function returns a string format for
# interval notation of the form: (-infty,d)  
def INEQ_NINF_INT(D,SL):
  global INEQ_NI_INT
  INEQ_NI_INT=""
  STR=""
  if (SR == 1):
    STR=")"
  else:
    STR="]" 
    
  INEQ_NI_INT= str(D) + STR 
  print(INEQ_NI_INT)
  return  INEQ_NI_INT  
       



# Generate question and answer lists for each problem type:
expression1=[INEQ_AB(a[i],b[i],INQL[i],INQR[i]) for i in range(len(a))]
answer1=[INEQ_INT(a[i],b[i],INQL[i],INQR[i]) for i in range(len(a))] 

expression2=[INEQ_PINF(c[i],INQL[i]) for i in range(len(a))]
answer2=[INEQ_PINF_INT(c[i],INQL[i]) for i in range(len(a))] 

expression3=[INEQ_NINF(d[i],INQR[i]) for i in range(len(a))]
answer3=[INEQ_NINF_INT(d[i],INQR[i]) for i in range(len(a))] 


# generate the latex code for the questions and answers
for i in range(len(a)):
  qoutput1 += r" \item $%s$" %(expression1[i])
  aoutput1 += r" \item $%s$" %(answer1[i])
  qoutput2 += r" \item $%s$" %(expression2[i])
  aoutput2 += r" \item $%s ,\infty  )$" %(answer2[i])
  qoutput3 += r" \item $%s$" %(expression3[i])
  aoutput3 += r" \item $(-\infty, %s$" %(answer3[i])

\end{sagesilent}








%%%%%%%%%%%%%%%%%%%%  Latex Output %%%%%%%%%%%%%%%%%%%%

% Random Seed can be used to re-create a particular worksheet
\noindent Write the following inequalities in interval notation: \\
\textcolor{gray}{Random Seed: $\sage{t}$}


% Create list of problems via qoutput
\begin{multicols}{3}
\begin{enumerate}
\sagestr{qoutput1}
\sagestr{qoutput2}
\sagestr{qoutput3}
\end{enumerate}
\end{multicols}


\newpage


% Create associated list of solutions via aoutput
\begin{multicols}{3}
\begin{enumerate}
\sagestr{aoutput1}
\sagestr{aoutput2}
\sagestr{aoutput3}
\end{enumerate}
\end{multicols}


\end{document}