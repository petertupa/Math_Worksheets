\documentclass{article}%
\usepackage{sagetex}
\usepackage{multicol}
\usepackage{tikz}
\usepackage{amsmath}
\setlength {\textwidth} {7.5in} 
\setlength {\textheight}{10.0in}
\setlength{\oddsidemargin} {-0.5in}
\setlength {\evensidemargin} {-0.5in}
\setlength{\topmargin} {-0.0in} 
\setlength {\headheight} {4.0pt} 
\setlength {\headsep} {0in}
\setlength{\columnsep}{0.5cm}


\begin{document}


\begin{sagesilent}
import random
x,y,z=var('x,y,z')
# Generate and set a random seed.
t=ZZ.random_element(999999)
set_random_seed(t)
# Generate lists of random integers to be used
a=[ZZ.random_element(2,14) for i in range(20)]
b=[ZZ.random_element(2,5) for i in range(len(a))]
c=[ZZ.random_element(2,15) for i in range(len(a))]
d=[(-1)^(ZZ.random_element(1,3))*ZZ.random_element(2,20) for i in range(len(a))]
f=[(-1)^(ZZ.random_element(1,3))*ZZ.random_element(1,8) for i in range(len(a))]
g=[(-1)^(ZZ.random_element(1,3))*ZZ.random_element(2,34) for i in range(len(a))]
h=[(-1)^(ZZ.random_element(1,3))*ZZ.random_element(1,12) for i in range(len(a))]
k=[(-1)^(ZZ.random_element(1,3))*ZZ.random_element(1,12) for i in range(len(a))]
m=[ZZ.random_element(1,4) for i in range(len(a))]
n=[ZZ.random_element(2,6) for i in range(len(a))]
p=[ZZ.random_element(1,4) for i in range(len(a))]
q=[ZZ.random_element(2,6) for i in range(len(a))]
u=[ZZ.random_element(1,4) for i in range(len(a))]
r=[ZZ.random_element(2,6) for i in range(len(a))]
s=[ZZ.random_element(2,6) for i in range(len(a))]

# List of possible fractions to use in exponents (only positive / both pos & neg)
FRAC=[1/2,1/2,3/2,1/3,2/3,1/4,3/4,1/5,2/5,3/5,4/5,2,3,4,5]  #16 options
FRAC2=[1/2,-1/2,3/2,-3/2,1/3,-1/3,2/3,-2/3,1/4,-1/4,3/4,-3/4,1/5,-1/5,2/5,-2/5,3/5,-3/5,4/5,-4/5,2,3,4,5]  #16 options

# Generate random selection of fraction list
RFRAC=[ZZ.random_element(0,15) for i in range(len(a))]
RFRAC2=[ZZ.random_element(0,25) for i in range(len(a))]
RFRAC3=[ZZ.random_element(0,15) for i in range(len(a))]

# Generate random list of prime numbers
PL=[2,3,5,7]
RPL=[PL[ZZ.random_element(0,4)] for i in range(len(a))]
RPL_EXP=[RPL[i]^n[i] for i in range(len(a))]
 
# Generate random list of bases
BASE=[2,3,4,5,6,7,x,x,x,y,y]
RBASE=[BASE[ZZ.random_element(0,10)] for i in range(len(a))]


# check to make sure exponential num/den are not the same
for j in range(len(a)):
  while m[j]==n[j]:
    m[j]=ZZ.random_element(1,4)
    n[j]=ZZ.random_element(2,6)

for j in range(len(a)):
  while p[j]==q[j]:
    p[j]=ZZ.random_element(1,4)
    q[j]=ZZ.random_element(2,6)
    
for j in range(len(a)):
  while u[j]==r[j]:
    u[j]=ZZ.random_element(1,4)
    r[j]=ZZ.random_element(2,6)    
    
    





# generate the solution of each problem 

# Problem Type1: x^m/n * x^p/q
answer1=[m[i]/n[i]+p[i]/q[i] for i in range(len(a))]
# Check if answer1 =1, set output to '' so nothing appears in LateX output
for i in range(len(a)):
  if answer1[i]==1:
    answer1[i]=''   

# Problem Type2:  x^(m/n) / x^(p/q)   
answer2=[p[i]/r[i]-m[i]/s[i] for i in range(len(a))]
   
# Problem Type3:  ( x^(m/n) )^(p/q)

# Problem Type4:  ( x^(m/n) * y^(p/q) )^k     
answer4=[(x^FRAC[RFRAC[i]]*y^FRAC2[RFRAC2[i]])^FRAC[RFRAC3[i]] for i in range(len(a))]
answer4=[answer4[i].canonicalize_radical() for i in range(len(a))]        #Force to simplify nicely
     
    
    
# assign variables to a null character
qoutput1=""
aoutput1=""
qoutput2=""
aoutput2=""
qoutput3=""
aoutput3=""
qoutput4=""
aoutput4=""


# generate the latex code for the questions and answers
for i in range(len(a)):
  qoutput1 += r" \item $%s^\frac{%s}{%s} \cdot %s^\frac{%s}{%s}$" %(latex(RBASE[i]),m[i],n[i],latex(RBASE[i]),p[i],q[i])
  aoutput1 += r" \item $%s^%s $" %(latex(RBASE[i]),latex(answer1[i]))
  
  qoutput2 += r" \item $\frac{%s^\frac{%s}{%s}}{%s^\frac{%s}{%s}}  $" %(latex(RBASE[i]),p[i],r[i],latex(RBASE[i]),m[i],s[i])
  if answer2[i] > 0:
    aoutput2 += r" \item $%s^%s$" %(latex(RBASE[i]),latex(answer2[i]))    #a>0 then x^a
  elif answer2[i] < 0:
    aoutput2 += r" \item $\frac{1}{%s^%s}$" %(latex(RBASE[i]),latex(abs(answer2[i])))  #a<0 then 1/x^a
  elif answer2[i] == 1:  
    aoutput2 += r" \item $%s $" %(latex(RBASE[i]))       #a=1 then x 
  elif answer2[i] == 0: 
    aoutput2 += r" \item $1 $"    #a=0 then 1

  qoutput3 += r" \item $(%s^\frac{%s}{%s})^\frac{%s}{%s} $" %(latex(RBASE[i]),u[i],r[i],m[i],n[i])
  aoutput3 += r" \item $%s$" %(latex(RBASE[i]^(u[i]*m[i]/r[i]/n[i])))

  qoutput4 += r" \item $(x^%s y^{%s})^%s $" %(latex(FRAC[RFRAC[i]]),latex(FRAC2[RFRAC2[i]]),latex(FRAC[RFRAC3[i]]))
  aoutput4+= r" \item $%s$" %(latex(answer4[i]))


\end{sagesilent}








%%%%%%%%%%%%%%%%%%%%  Latex Output %%%%%%%%%%%%%%%%%%%%

% Random Seed can be used to re-create a particular worksheet
\noindent Simplify using the properties of rational exponents: \\
\textcolor{gray}{Random Seed: $\sage{t}$}


% Create list of problems via qoutput
\begin{enumerate}
\begin{multicols}{3}
\sagestr{qoutput1}
\sagestr{qoutput2}
\sagestr{qoutput3}
\sagestr{qoutput4}
\end{multicols}
\end{enumerate}


\newpage


% Create associated list of solutions via aoutput
\begin{multicols}{3}
\begin{enumerate}
\sagestr{aoutput1}
\sagestr{aoutput2}
\sagestr{aoutput3}
\sagestr{aoutput4}
\end{enumerate}
\end{multicols}


\end{document}