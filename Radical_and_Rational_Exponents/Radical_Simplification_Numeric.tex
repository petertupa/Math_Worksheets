\documentclass{article}%
\usepackage{sagetex}
\usepackage{multicol}
\usepackage{tikz}
\usepackage{amsmath}
\setlength {\textwidth} {7.5in} 
\setlength {\textheight}{10.0in}
\setlength{\oddsidemargin} {-0.5in}
\setlength {\evensidemargin} {-0.5in}
\setlength{\topmargin} {-0.0in} 
\setlength {\headheight} {4.0pt} 
\setlength {\headsep} {0in}
\setlength{\columnsep}{0.5cm}


\begin{document}


\begin{sagesilent}
import random
# Generate and set a random seed.
t=ZZ.random_element(999999)
set_random_seed(t)
# Generate lists of random integers to be used
a=[ZZ.random_element(2,14) for i in range(20)]
b=[(-1)^(ZZ.random_element(1,3))*ZZ.random_element(2,11) for i in range(len(a))]
c=[ZZ.random_element(200,1000) for i in range(len(a))]
d=[(-1)^(ZZ.random_element(1,3))*ZZ.random_element(2,20) for i in range(len(a))]
f=[(-1)^(ZZ.random_element(1,3))*ZZ.random_element(1,8) for i in range(len(a))]
g=[(-1)^(ZZ.random_element(1,3))*ZZ.random_element(2,34) for i in range(len(a))]
h=[(-1)^(ZZ.random_element(1,3))*ZZ.random_element(1,12) for i in range(len(a))]
k=[(-1)^(ZZ.random_element(1,3))*ZZ.random_element(1,12) for i in range(len(a))]
m=[ZZ.random_element(3,6) for i in range(len(a))]
REXP=[ZZ.random_element(0,2) for i in range(len(a))]  #Switch to turn on/off variable term
RPN=[(-1)^(ZZ.random_element(1,3)) for i in range(len(a))]  #Random positive/negative

#Matrix with three rows, each with reasonable random roots for different randical indices
RMAT2=matrix(ZZ,3,len(a))
for i in range(len(a)):
  RMAT2[0,i]=ZZ.random_element(1,12)
  RMAT2[1,i]=ZZ.random_element(1,5)
  RMAT2[2,i]=ZZ.random_element(1,4)

print(RMAT2)  
  
# generate the question and solution of each equation

# Problem Type1: sqrt(a)=b
expression1=[sqrt(SR(a[i]^2),hold=True) for i in range(len(a))]
answer1=[a[i] for i in range(len(a))]


# Problem Type2: \sqrt[n](+/- ax^m)=bx
expression2=[(RPN[i]*(RMAT2[m[i]-3,i]*x^REXP[i])^m[i]) for i in range(len(a))]
answer2=[RPN[i]*RMAT2[m[i]-3,i]*x^REXP[i] for i in range(len(a))]


# assign variables to a null character
qoutput1=""
aoutput1=""
qoutput2=""
aoutput2=""


# generate the latex code for the questions and answers
for i in range(len(a)):
  qoutput1 += r" \item $%s$" %(latex(expression1[i]))
  qoutput2 += r" \item $\sqrt[%s]{%s}$" %(latex(m[i]),latex(expression2[i]))
  aoutput1 += r" \item $%s$" %(latex(answer1[i]))
  if (m[i] %2 ==0 and RPN[i] > 0 and REXP[i]>0 ):    #With variable and even index gets AbsVal
    aoutput2 += r" \item $|%s|$" %(latex(answer2[i]))
  elif (m[i] %2 ==0 and RPN[i] < 0 ):                #negative argument and even index  is not real
    aoutput2 += r" \item $\rm{N.A.R.N.}$" 
  else:                                              # Otherwise doable
    aoutput2 += r" \item $%s$" %(latex(answer2[i]))


\end{sagesilent}









%%%%%%%%%%%%%%%%%%%%  Latex Output %%%%%%%%%%%%%%%%%%%%

% Random Seed can be used to re-create a particular worksheet
\noindent Simplify or state that the expression is not a real number: \\
\textcolor{gray}{Random Seed: $\sage{t}$}


% Create list of problems via qoutput
\begin{multicols}{3}
\begin{enumerate}
\sagestr{qoutput1}
\sagestr{qoutput2}
\end{enumerate}
\end{multicols}


\newpage


% Create associated list of solutions via aoutput
\begin{multicols}{3}
\begin{enumerate}
\sagestr{aoutput1}
\sagestr{aoutput2}
\end{enumerate}
\end{multicols}


\end{document}