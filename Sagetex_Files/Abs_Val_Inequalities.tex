\documentclass{article}%
\usepackage{sagetex}
\usepackage{multicol}
\usepackage{tikz}
\usepackage{amsmath}
\setlength {\textwidth} {7.5in} 
\setlength {\textheight}{10.0in}
\setlength{\oddsidemargin} {-0.5in}
\setlength {\evensidemargin} {-0.5in}
\setlength{\topmargin} {-0.0in} 
\setlength {\headheight} {4.0pt} 
\setlength {\headsep} {0in}
\setlength{\columnsep}{0.5cm}


\begin{document}


\begin{sagesilent}
import random
# Generate and set a random seed.
tt=ZZ.random_element(999999)
set_random_seed(tt)
# Generate lists of random integers to be used
PM=[-1,1,1,1,1,1]   # Used to choose sign of constant outside of absolute value
a=[(-1)^(ZZ.random_element(1,3))*ZZ.random_element(1,20) for i in range(10)]
b=[(-1)^(ZZ.random_element(1,3))*ZZ.random_element(1,20) for i in range(len(a))]
c=[PM[(ZZ.random_element(0,6))]*ZZ.random_element(1,10) for i in range(len(a))]
d=[(-1)^(ZZ.random_element(1,3))*ZZ.random_element(2,10) for i in range(len(a))]
f=[(-1)^(ZZ.random_element(1,3))*ZZ.random_element(1,10) for i in range(len(a))]
g=[ZZ.random_element(2,10) for i in range(len(a))]
h=[ZZ.random_element(1,12) for i in range(len(a))]
k=[(-1)^(ZZ.random_element(1,3))*ZZ.random_element(1,12) for i in range(len(a))]
m=[(-1)^(ZZ.random_element(1,3))*ZZ.random_element(1,12) for i in range(len(a))]
n=[(-1)^(ZZ.random_element(1,3))*ZZ.random_element(1,12) for i in range(len(a))]
p=[(-1)^(ZZ.random_element(1,3))*ZZ.random_element(1,12) for i in range(len(a))]
q=[(-1)^(ZZ.random_element(1,3))*ZZ.random_element(1,12) for i in range(len(a))]
r=[PM[(ZZ.random_element(0,6))]*ZZ.random_element(6,20) for i in range(len(a))]
s=[ZZ.random_element(2,12) for i in range(len(a))]
t=[(-1)^(ZZ.random_element(1,3))*ZZ.random_element(1,12) for i in range(len(a))]
v=[(-1)^(ZZ.random_element(1,3))*ZZ.random_element(1,12) for i in range(len(a))]
w=[ZZ.random_element(2,8) for i in range(len(a))]


# Setup list of inequality types and random choice of sign for each index
# Note Indices of i:  0 for <, 1 for <=, 2 for >, 3 for >=
LINALT=('<','<=','>','>=')
LIN=['<','\leq','>','\geq',]


# check for conditions that would result in incorrect inequality
for j in range(len(a)):
  while a[j]==b[j]:
    a[j]=(-1)^(ZZ.random_element(1,3))*ZZ.random_element(1,20)
    b[j]=(-1)^(ZZ.random_element(1,3))*ZZ.random_element(1,20)

for j in range(len(a)):
  while abs(h[j])==abs(g[j]):
    h[j]=(-1)^(ZZ.random_element(1,3))*ZZ.random_element(1,20)
    g[j]=(-1)^(ZZ.random_element(1,3))*ZZ.random_element(1,20)    

for j in range(len(a)):
  while abs(w[j])>=abs(r[j]):
    r[j]=PM[(ZZ.random_element(0,6))]*ZZ.random_element(6,20)
    w[j]=ZZ.random_element(2,8) 

    
# Returns string formated solution for two AND inequalities  
# Assumes input values, A & B, are in solved forms
# A is associated with SL inequality sign 
# B is associated with SR inequality sign 
def INEQ_INTERSECTION(A,B,SL,SR):
  global INEQ_INT
  INEQ_INT=""
  if SL>=2 and SR<=1:
    if A>B: STR="\emptyset"
    elif SL==2 and SR==0:   STR="(" + str(A) + "," + str(B) + ")"
    elif SL==3 and SR==0:   STR="[" + str(A) + "," + str(B) + ")"
    elif SL==2 and SR==1:   STR="(" + str(A) + "," + str(B) + "]"
    elif SL==3 and SR==1:   STR="[" + str(A) + "," + str(B) + "]"
    else: STR="Err: INEQ_INTERSECTION: 1st IF branch"   
  elif SL<=1 and SR>=2:
    if A<B: STR="\emptyset"
    elif SL==0 and SR==2:   STR="(" + str(B) + "," + str(A) + ")" 
    elif SL==1 and SR==2:   STR="(" + str(B) + "," + str(A) + "]"
    elif SL==0 and SR==3:   STR="[" + str(B) + "," + str(A) + ")"
    elif SL==1 and SR==3:   STR="[" + str(B) + "," + str(A) + "]"    
    else: STR="Err: INEQ_INTERSECTION: 2nd IF branch"   
  elif SL<=1 and SR<=1:
    if A>B:
      if SR==0:   STR="(-\infty," + str(B) + ")"
      elif SR==1:   STR="(-\infty," + str(B) + "]"
    elif A<B:
      if SL==0:   STR="(-\infty," + str(A) + ")"
      elif SL==1:   STR="(-\infty," + str(A) + "]"      
    else: STR="Err: INEQ_INTERSECTION: 3rd IF branch"  
  elif SL>=2 and SR>=2:
    if A>B:
      if SL==2:     STR="(" + str(A) + ",\infty)"
      elif SL==3:   STR="[" + str(A) + ",\infty)"
    elif A<B:
      if SR==2:     STR="(" + str(B) + ",\infty)"
      elif SR==3:   STR="[" + str(B) + ",\infty)"
    else: STR="Err: INEQ_INTERSECTION: 4th IF branch"
  else:
    STR="Err: INEQ_INTERSECTION: Missed all IFs"
    
  INEQ_INT=STR
  print(INEQ_INT)
  return INEQ_INT

# Returns string formated solution for two OR inequalities  
# Assumes input values, A & B, are in solved forms
# A is associated with SL inequality sign 
# B is associated with SR inequality sign  
def INEQ_UNION(A,B,SL,SR):
  global INEQ_UN
  INEQ_UN=""
  if SL>=2 and SR<=1:
    if A<B: STR="(-\infty,\infty)"
    elif SL==2 and SR==0:   STR="(-\infty," + str(B) + ") \cup (" + str(A) + ",\infty)"
    elif SL==3 and SR==0:   STR="(-\infty," + str(B) + ") \cup [" + str(A) + ",\infty)"
    elif SL==2 and SR==1:   STR="(-\infty," + str(B) + "] \cup (" + str(A) + ",\infty)"
    elif SL==3 and SR==1:   STR="(-\infty," + str(B) + "] \cup [" + str(A) + ",\infty)"
    else: STR="Err: INEQ_UNION: 1st IF branch"   
  elif SL<=1 and SR>=2:
    if A>B: STR="(-\infty,\infty)"
    elif SL==0 and SR==2:   STR="(-\infty," + str(A) + ") \cup (" + str(B) + ",\infty)" 
    elif SL==1 and SR==2:   STR="(-\infty," + str(A) + "] \cup (" + str(B) + ",\infty)"
    elif SL==0 and SR==3:   STR="(-\infty," + str(A) + ") \cup [" + str(B) + ",\infty)"
    elif SL==1 and SR==3:   STR="(-\infty," + str(A) + "] \cup [" + str(B) + ",\infty)"    
    else: STR="Err: INEQ_UNION: 2nd IF branch"   
  elif SL<=1 and SR<=1:
    if A>B:
      if SL==0:   STR="(-\infty," + str(A) + ")"
      elif SL==1:   STR="(-\infty," + str(A) + "]"
    elif A<B:
      if SR==0:   STR="(-\infty," + str(B) + ")"
      elif SR==1:   STR="(-\infty," + str(B) + "]"      
    else: STR="Err: INEQ_UNION: 3rd IF branch"  
  elif SL>=2 and SR>=2:
    if A>B:
      if SR==2:     STR="(" + str(B) + ",\infty)"
      elif SR==3:   STR="[" + str(B) + ",\infty)"
    elif A<B:
      if SL==2:     STR="(" + str(A) + ",\infty)"
      elif SL==3:   STR="[" + str(A) + ",\infty)"
    else: STR="Err: INEQ_UNION: 4th IF branch"
  else:
    STR="Err: INEQ_UNION: Missed all IFs"
    
  INEQ_UN=STR
  print(INEQ_UN)
  return INEQ_UN 

 

 
# Returns opposite sign inequality of same type
# Use for setting up paired inequality
# or when mulitplication by a negative occurs
def INEQ_SWTCH(A):
  global INEQPR
  if   A == 0:    INEQSW = 2
  elif A == 1:    INEQSW = 3
  elif A == 2:    INEQSW = 0
  elif A == 3:    INEQSW = 1
  
  return INEQSW
  
  
  
# Returns string formated solution for |u|=v  
# Assumes input for both +(P) / -(M) in solved forms
# NC = Negative Check for constants outside abs.val
def ABS_EQN(P,M,NC):
  global ABEQ
  ABEQ=""
  if NC<0:
    ABEQ="\emptyset"
  elif NC==0:
    ABEQ="\{" + str(P) + "\}"
  elif P<M:
    ABEQ="\{" + str(P) + "," + str(M) + "\}"
  else:
    ABEQ="\{" + str(M) + "," + str(P) + "\}"
  
  print(ABEQ)
  return ABEQ
  
    

# generate the solution of each problem 
# Problem Type1: |ax+b|=c
# Problem Type2: |ax+b|+c=d
# Problem Type3: |ax+b|=|cx+d|
answer1=[ABS_EQN(  (c[i]-b[i])/a[i] , (-c[i]-b[i])/a[i] , c[i] )   for i in range(len(a))]
answer2=[ABS_EQN(  (h[i]-g[i]-f[i])/d[i] , (g[i]-h[i]-f[i])/d[i] , h[i]-g[i] )   for i in range(len(a))]
answer3=[ABS_EQN(  (b[i]-k[i])/(h[i]-g[i]) , (-k[i]-b[i])/(h[i]+g[i]) , 1 )   for i in range(len(a))]




######Setup for Problem Type4###########    
######      |ax+b| > c       ###########

# R_andom IN_equality Q_uestion/A_nswer Type# A(left)/B(right)
RINQ4A=[ZZ.random_element(0,4) for i in range(len(a))] #Initial random inequality switch
RINQ4B=[INEQ_SWTCH( RINQ4A[i]) for i in range(len(a))] #Paired inequality  
RINA4A=[RINQ4A[i] for i in range(len(a))]              # inequality for answer, copied initially
RINA4B=[RINQ4B[i] for i in range(len(a))]              # inequality for answer, copied initially

#Check if a sign change needed and update the ineq signs for answers
for i in range(len(a)):
  if p[i]<0:
    RINA4A[i]=INEQ_SWTCH( RINQ4A[i])
    RINA4B[i]=INEQ_SWTCH( RINQ4B[i])
   
# Check conditions to see which type of solution process is needed    
answer4=[x for i in range(len(a))]
for i in range(len(a)):
  if RINQ4A[i]<=1 and c[i]<0:
    answer4[i]="\emptyset"
  elif RINQ4A[i]>=2 and c[i]<0:
    answer4[i]="(-\infty,\infty)"
  elif RINQ4A[i]<=1 and c[i]>0: 
    answer4[i]=INEQ_INTERSECTION(  (c[i]-q[i])/p[i]  ,  (-c[i]-q[i])/p[i]  ,RINA4A[i],RINA4B[i])
  elif RINQ4A[i]>=2 and c[i]>0: 
    answer4[i]=INEQ_UNION(  (c[i]-q[i])/p[i]  ,  (-c[i]-q[i])/p[i]  ,RINA4A[i],RINA4B[i])  

    
    
    
######Setup for Question Type 5###########  
######      a|bx+c|+d > f      ########### 
RINQ5A=[ZZ.random_element(0,4) for i in range(len(a))]
RINQ5B=[INEQ_SWTCH( RINQ5A[i]) for i in range(len(a))]
RINA5A=[RINQ5A[i] for i in range(len(a))]
RINA5B=[RINQ5B[i] for i in range(len(a))]

for i in range(len(a)):
  if t[i]<0:
    RINA5A[i]=INEQ_SWTCH( RINQ5A[i])
    RINA5B[i]=INEQ_SWTCH( RINQ5B[i])

    
#Setup abs_val coefficient as a divisor of collected terms on right:
Stemp=[r[i]-w[i]  for i in range(len(a))]
S=[divisors(Stemp[i])[ZZ.random_element(1,len(divisors(Stemp[i])))]  for i in range(len(a))]
print(S) 
 
answer5=[x for i in range(len(a))]
for i in range(len(a)):
  if RINQ5A[i]<=1 and (r[i]-w[i])/S[i]<0:
    answer5[i]="\emptyset"
  elif RINQ5A[i]>=2 and (r[i]-w[i])/S[i]<0:
    answer5[i]="(-\infty,\infty)"
  elif RINQ5A[i]<=1 and (r[i]-w[i])/S[i]>0: 
    answer5[i]=INEQ_INTERSECTION(  ( r[i]-w[i]-S[i]*v[i] )/(S[i]*t[i] )   ,  (w[i]-r[i]-S[i]*v[i] )/(S[i]*t[i])  ,RINA5A[i],RINA5B[i])
  elif RINQ5A[i]>=2 and (r[i]-w[i])/S[i]>0: 
    answer5[i]=INEQ_UNION( (r[i]-w[i]-S[i]*v[i] )/(S[i]*t[i])  ,  (w[i]-r[i]-S[i]*v[i] )/(S[i]*t[i])  ,RINA5A[i],RINA5B[i])     
   
   
   
   
############ Setup for question type 6 ##################
####        | (ax+b)/c | > d

RINQ6A=[ZZ.random_element(0,4) for i in range(len(a))]
RINQ6B=[INEQ_SWTCH( RINQ6A[i]) for i in range(len(a))]
RINA6A=[RINQ6A[i] for i in range(len(a))]
RINA6B=[RINQ6B[i] for i in range(len(a))]


for i in range(len(a)):
  if n[i]<0:
    RINA6A[i]=INEQ_SWTCH( RINQ6A[i])
    RINA6B[i]=INEQ_SWTCH( RINQ6B[i])
    
answer6=[x for i in range(len(a))]
for i in range(len(a)):
  if RINQ6A[i]<=1 and c[i]<0:
    answer6[i]="\emptyset"
  elif RINQ6A[i]>=2 and c[i]<0:
    answer6[i]="(-\infty,\infty)"
  elif RINQ6A[i]<=1 and c[i]>0: 
    answer6[i]=INEQ_INTERSECTION(  ( s[i]*c[i]-m[i] )/(n[i] )   ,  ( -s[i]*c[i]-m[i] )/(n[i] )  ,RINA6A[i],RINA6B[i])
  elif RINQ6A[i]>=2 and c[i]>0: 
    answer6[i]=INEQ_UNION( ( s[i]*c[i]-m[i] )/(n[i] )  ,  ( -s[i]*c[i]-m[i] )/(n[i] )  ,RINA6A[i],RINA6B[i])     
    
    
    
# assign variables to a null character
qoutput1=""
aoutput1=""
qoutput2=""
aoutput2=""
qoutput3=""
aoutput3=""
qoutput4=""
aoutput4=""
qoutput5=""
aoutput5=""
qoutput6=""
aoutput6=""

# generate the latex code for the questions and answers
for i in range(len(a)):
  qoutput1 += r" \item $ | %s | = %s$" %( latex(a[i]*x+b[i]),c[i] )
  qoutput2 += r" \item $ | %s | + %s = %s$" %( latex(d[i]*x+f[i]),g[i],h[i] )
  qoutput3 += r" \item $ | %s |  = | %s | $" %( latex(h[i]*x+k[i]),latex(g[i]*x+b[i]) )
  qoutput4 += r" \item $ | %s | %s %s $" %( latex(p[i]*x+q[i]), LIN[RINQ4A[i]], c[i] )
  qoutput5 += r" \item $ %s | %s | + %s %s %s $" %(S[i], latex(t[i]*x+v[i]), w[i] ,  LIN[RINQ5A[i]], r[i] )
  qoutput6 += r" \item $ | \frac{%s}{%s} | %s %s $" %(latex(n[i]*x+m[i]), s[i] ,  LIN[RINQ6A[i]], c[i] )
  aoutput1 += r" \item $%s$" %( answer1[i] )
  aoutput2 += r" \item $%s$" %( answer2[i] )
  aoutput3 += r" \item $%s$" %( answer3[i] )
  aoutput4 += r" \item $%s$" %( answer4[i] )
  aoutput5 += r" \item $%s$" %( answer5[i] )
  aoutput6 += r" \item $%s$" %( answer6[i] )

  
\end{sagesilent}





%%%%%%%%%%%%%%%%%%%%  Latex Output %%%%%%%%%%%%%%%%%%%%

% Random Seed can be used to re-create a particular worksheet
\noindent Solve each equation or inequality. Except for the empty set, express the solution set in interval notation: \\
\textcolor{gray}{Random Seed: $\sage{tt}$}


% Create list of problems via qoutput
\begin{multicols}{3}
\begin{enumerate}
\sagestr{qoutput1}
\sagestr{qoutput2}
\sagestr{qoutput3}
\sagestr{qoutput4}
\sagestr{qoutput5}
\sagestr{qoutput6}
\end{enumerate}
\end{multicols}


\newpage


% Create associated list of solutions via aoutput
\begin{multicols}{3}
\begin{enumerate}
\sagestr{aoutput1}
\sagestr{aoutput2}
\sagestr{aoutput3}
\sagestr{aoutput4}
\sagestr{aoutput5}
\sagestr{aoutput6}
\end{enumerate}
\end{multicols}


\end{document}