\documentclass{article}%
\usepackage{sagetex}
\usepackage{multicol}
\usepackage{tikz}
\usepackage{amsmath}
\setlength {\textwidth} {7.5in} 
\setlength {\textheight}{10.0in}
\setlength{\oddsidemargin} {-0.5in}
\setlength {\evensidemargin} {-0.5in}
\setlength{\topmargin} {-0.0in} 
\setlength {\headheight} {4.0pt} 
\setlength {\headsep} {0in}
\setlength{\columnsep}{0.5cm}


\begin{document}


\begin{sagesilent}
import random
# Generate and set a random seed.
t=ZZ.random_element(999999)
set_random_seed(t)
# Generate lists of random integers to be used
a=[(-1)^(ZZ.random_element(1,3))*ZZ.random_element(1,10) for i in range(12)]
b=[(-1)^(ZZ.random_element(1,3))*ZZ.random_element(1,15) for i in range(len(a))]
c=[(-1)^(ZZ.random_element(1,3))*ZZ.random_element(1,8) for i in range(len(a))]
d=[(-1)^(ZZ.random_element(1,3))*ZZ.random_element(2,20) for i in range(len(a))]
f=[(-1)^(ZZ.random_element(1,3))*ZZ.random_element(1,8) for i in range(len(a))]
g=[(-1)^(ZZ.random_element(1,3))*ZZ.random_element(2,34) for i in range(len(a))]
h=[(-1)^(ZZ.random_element(1,3))*ZZ.random_element(1,12) for i in range(len(a))]
k=[(-1)^(ZZ.random_element(1,3))*ZZ.random_element(1,12) for i in range(len(a))]
m=[ZZ.random_element(1,10) for i in range(len(a))]
n=[(-1)^(ZZ.random_element(1,3))*ZZ.random_element(1,10) for i in range(len(a))]
q=[ZZ.random_element(1,10) for i in range(len(a))]
r=[(-1)^(ZZ.random_element(1,3))*ZZ.random_element(1,10) for i in range(len(a))]


# check for conditions that would result in a difference of squares
for j in range(len(a)):
  while a[j]==-b[j]:
    a[j]=(-1)^(ZZ.random_element(1,3))*(ZZ.random_element(1,10))
    b[j]=(-1)^(ZZ.random_element(1,3))*(ZZ.random_element(1,15))



# Setup polynomials and store these each in a list
expression1=[expand((x+a[i])*(x+b[i])) for i in range(len(a))]                              #polynomial of the form: (x+a)(x+b) 
expression2=[expand((c[i]*x+d[i])*(f[i]*x+g[i])) for i in range(len(a))]                    #polynomial of the form: (ax+b)(cx+d)
expression3=[expand((h[i]*x+k[i])*(h[i]*x-k[i])) for i in range(len(a))]                    #polynomial of the form: (ax+b)(ax-b) = (ax)^2-b^2
expression4=[expand((m[i]*x+n[i])*(m[i]^2*x^2-m[i]*n[i]*x+n[i]^2)) for i in range(len(a))]  #polynomial of the form: (ax+b)((ax)^2-abx+b^2) = (ax)^3 +/- b^3 
expression5=[expand((q[i]*x+r[i])^2) for i in range(len(a))]                                #polynomial of the form: (ax)^2+2abx+b^2 = (ax+b)^2


# generate the solution of each factorization
answer1=[factor(expression1[i]) for i in range(len(a))]
answer2=[factor(expression2[i]) for i in range(len(a))]
answer3=[factor(expression3[i]) for i in range(len(a))]
answer4=[factor(expression4[i]) for i in range(len(a))]
answer5=[factor(expression5[i]) for i in range(len(a))]


# assign variables to a null character
qoutput1=""
qoutput2=""
qoutput3=""
qoutput4=""
qoutput5=""
aoutput1=""
aoutput2=""
aoutput3=""
aoutput4=""
aoutput5=""


# generate the latex code for the questions and answers
for i in range(len(a)):
  qoutput1 += r" \item $%s$" %(latex(expression1[i]))
  qoutput2 += r" \item $%s$" %(latex(expression2[i]))
  qoutput3 += r" \item $%s$" %(latex(expression3[i]))
  qoutput4 += r" \item $%s$" %(latex(expression4[i]))
  qoutput5 += r" \item $%s$" %(latex(expression5[i]))
  aoutput1 += r" \item $%s$" %(latex(answer1[i]))
  aoutput2 += r" \item $%s$" %(latex(answer2[i])) 
  aoutput3 += r" \item $%s$" %(latex(answer3[i])) 
  aoutput4 += r" \item $%s$" %(latex(answer4[i])) 
  aoutput5 += r" \item $%s$" %(latex(answer5[i])) 


\end{sagesilent}


%%%%%%%%%%%%%%%%%%%%  Latex Output %%%%%%%%%%%%%%%%%%%%

% Random Seed can be used to re-create a particular worksheet
\noindent Factor the given trinomial. \\
\textcolor{gray}{Random Seed: $\sage{t}$}


% Create list of problems via qoutput
\begin{multicols}{3}
\begin{enumerate}
\sagestr{qoutput1}
\sagestr{qoutput2}
\sagestr{qoutput3}
\sagestr{qoutput4}
\sagestr{qoutput5}
\end{enumerate}
\end{multicols}


\vfill
\newpage


% Create associated list of solutions via aoutput
\begin{multicols}{3}
\begin{enumerate}
\sagestr{aoutput1}
\sagestr{aoutput2}
\sagestr{aoutput3}
\sagestr{aoutput4}
\sagestr{aoutput5}
\end{enumerate}
\end{multicols}





\end{document}