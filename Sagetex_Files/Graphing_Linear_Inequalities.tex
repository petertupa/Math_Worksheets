\documentclass{article}%
\usepackage{sagetex}
\usepackage{multicol}
\usepackage{tikz}
\usepackage{amsmath}
\setlength {\textwidth} {7.5in} 
\setlength {\textheight}{10.0in}
\setlength{\oddsidemargin} {-0.5in}
\setlength {\evensidemargin} {-0.5in}
\setlength{\topmargin} {-0.0in} 
\setlength {\headheight} {4.0pt} 
\setlength {\headsep} {0in}
\setlength{\columnsep}{0.5cm}


\begin{document}


\begin{sagesilent}
import random
# Generate and set a random seed.
t=ZZ.random_element(999999)
set_random_seed(t)
# Generate lists of random integers to be used
a=[(-1)^(ZZ.random_element(1,3))*ZZ.random_element(0,9) for i in range(6)]
b=[(-1)^(ZZ.random_element(1,3))*ZZ.random_element(1,9) for i in range(len(a))]
c=[(-1)^(ZZ.random_element(1,3))*ZZ.random_element(1,15) for i in range(len(a))]
d=[(-1)^(ZZ.random_element(1,3))*ZZ.random_element(0,9) for i in range(len(a))]
f=[(-1)^(ZZ.random_element(1,3))*ZZ.random_element(1,9) for i in range(len(a))]
g=[(-1)^(ZZ.random_element(1,3))*ZZ.random_element(1,15) for i in range(len(a))]
h=[(-1)^(ZZ.random_element(1,3))*ZZ.random_element(1,12) for i in range(len(a))]
k=[(-1)^(ZZ.random_element(1,3))*ZZ.random_element(1,12) for i in range(len(a))]
m=[ZZ.random_element(1,10) for i in range(len(a))]
n=[(-1)^(ZZ.random_element(1,3))*ZZ.random_element(1,10) for i in range(len(a))]
p=[(-1)^(ZZ.random_element(1,3))*ZZ.random_element(0,9) for i in range(len(a))]
q=[(-1)^(ZZ.random_element(1,3))*ZZ.random_element(1,9) for i in range(len(a))]
r=[(-1)^(ZZ.random_element(1,3))*ZZ.random_element(1,15) for i in range(len(a))]

x,y=var('x,y')           # Declare variables

# Setup list of inequality types and random choice of sign for each index
# Note Indices of i:  0 for <, 1 for <=, 2 for >, 3 for >=
# LS for linestyle: - for solid, -- for dashed
LINALT=('<','<=','>','>=')
LIN=['<','\leq','>','\geq']
LS=('--','-','--','-')   # 

# R_andom IN_equality Q_uestion/A_nswer Type# A(left)/B(right)
RINQ1=[ZZ.random_element(0,4) for i in range(len(a))] 	#Initial random inequality switch
RINA1=[RINQ1[i] for i in range(len(a))]                   #Inequality switch after accounting for negative y-coefficient
RINQ2A=[ZZ.random_element(0,4) for i in range(len(a))]
RINQ2B=[ZZ.random_element(0,4) for i in range(len(a))]
RINA2A=[RINQ2A[i] for i in range(len(a))]
RINA2B=[RINQ2B[i] for i in range(len(a))]


# Make all y-int nice integers
for j in range(len(a)):
  while b[j] != 0 and (c[j]/b[j]).is_integer() == False:
    b[j]=(-1)^(ZZ.random_element(1,3))*(ZZ.random_element(1,15))
    c[j]=(-1)^(ZZ.random_element(1,3))*(ZZ.random_element(1,15))

for j in range(len(a)):
  while f[j] != 0 and (g[j]/f[j]).is_integer() == False:
    f[j]=(-1)^(ZZ.random_element(1,3))*(ZZ.random_element(1,15))
    g[j]=(-1)^(ZZ.random_element(1,3))*(ZZ.random_element(1,15))
 
for j in range(len(a)):
  while q[j] != 0 and (r[j]/q[j]).is_integer() == False:
    q[j]=(-1)^(ZZ.random_element(1,3))*(ZZ.random_element(1,15))
    r[j]=(-1)^(ZZ.random_element(1,3))*(ZZ.random_element(1,15))
 
 
# Returns opposite sign inequality of same type
# Use for setting up paired inequality
# or when mulitplication by a negative occurs
def INEQ_SWTCH(A):
  global INEQPR
  if   A == 0:    INEQSW = 2
  elif A == 1:    INEQSW = 3
  elif A == 2:    INEQSW = 0
  elif A == 3:    INEQSW = 1
  
  return INEQSW 
 
 
 
 
# Setup expressions for questions of the form Ax+By and solved expressions
exp1=[ a[i]*x+b[i]*y     for i in range(len(a))]
ans1=[(-a[i]/b[i]*x+c[i]/b[i]) if b[i]!=0 else (10^4*x+10^4*c[i]) for i in range(len(a))]

exp2A=[ d[i]*x+f[i]*y     for i in range(len(a))]
ans2A=[-d[i]/f[i]*x+g[i]/f[i] for i in range(len(a))]
exp2B=[ p[i]*x+q[i]*y     for i in range(len(a))]
ans2B=[-p[i]/q[i]*x+r[i]/q[i] for i in range(len(a))]
  
    
# assign variables to a null character
qoutput1=""
qoutput2=""


# generate the latex code for the questions qoutput
for i in range(len(a)):
  qoutput1 += r" \item $%s %s %s   $"  %( latex( exp1[i]), LIN[RINQ1[i]]  , latex(c[i]) )
  qoutput2 += r" \item $\begin{cases} %s %s %s \\%s %s %s \end{cases} $" %(latex(exp2A[i]), LIN[RINQ2A[i]], latex(g[i]), latex(exp2B[i]), LIN[RINQ2B[i]], latex(r[i])   )


  
#Account for ineq sign switch via negative y-coefficient 
for i in range(len(a)):
  if b[i] < 0:
    RINA1[i]=INEQ_SWTCH( RINQ1[i])
    
  if f[i] < 0:
    RINA2A[i]=INEQ_SWTCH( RINQ2A[i])

  if q[i] < 0:
    RINA2B[i]=INEQ_SWTCH( RINQ2B[i])
          
    
    
#Set fill to shade above or below    
FILL1=[10 for i in range(len(a))]
FILL2A=[10 for i in range(len(a))]
FILL2B=[10 for i in range(len(a))]
for i in range(len(a)):
  if RINA1[i] == 0 or RINA1[i] == 1:
      FILL1[i] = -10
  if RINA2A[i] == 0 or RINA2A[i] == 1:
      FILL2A[i] = -10
  if RINA2B[i] == 0 or RINA2B[i] == 1:
      FILL2B[i] = -10
      
      
\end{sagesilent}





%%%%%%%%%%%%%%%%%%%%  Latex Output %%%%%%%%%%%%%%%%%%%%

% Random Seed can be used to re-create a particular worksheet
\noindent Solve each linear inequality and graph: \\
\textcolor{gray}{Random Seed: $\sage{t}$}



% Create list of problems via qoutput
\begin{multicols}{2}
\begin{enumerate}
\sagestr{qoutput1}
\sagestr{qoutput2}
\end{enumerate}



\newpage



\end{multicols}

\noindent Solutions:  \hfill (Note: \texttt{>=} and \texttt{<=} are equivalent to $\geq$ and $\leq$, respectively 



% Give LateX solution and appropriate inequality, then graph with shaded solution set and dotted/dashed line
% Students should know that only darkest region is full solution for the system of linear inequalities
% Was unable to get LateX so see sage list of inequality latex code
\begin{multicols}{3}
\begin{enumerate}
\foreach \i in {0,1,...,5}
{
 \item $y \:  \mathrm{ \sage{latex(LINALT[RINA1[\i]])} } \: \sage{latex(ans1[\i])}$ \\
\sageplot[scale=.35]{plot(ans1[\i], (x, -7, 7), gridlines='major',linestyle=LS[RINA1[\i]], ymin=-7, ymax=7, fill=FILL1[\i], fillcolor='grey', fillalpha=1/3)} 
}
\foreach \i in {0,1,...,5}
{
 \item $\begin{cases} y \:  \mathrm{ \sage{latex(LINALT[RINA2A[\i]])} } \: \sage{latex(ans2A[\i])}     \\  y \:  \mathrm{ \sage{latex(LINALT[RINA2B[\i]])} } \: \sage{latex(ans2B[\i])}     \\     \end{cases}$\\
\sageplot[scale=.3]{  plot(ans2A[\i], (x, -7, 7), gridlines='major',linestyle=LS[RINA2A[\i]], ymin=-7, ymax=7, fill=FILL2A[\i], fillcolor='grey', fillalpha=1/3) + plot(ans2B[\i], (x, -7, 7), gridlines='major',linestyle=LS[RINA2B[\i]], ymin=-7, ymax=7, fill=FILL2B[\i], fillcolor='grey', fillalpha=1/3) } 
}
\end{enumerate}
\end{multicols}





\end{document}