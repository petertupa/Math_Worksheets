\documentclass{article}%
\usepackage{sagetex}
\usepackage{multicol}
\usepackage{tikz}
\usepackage{amsmath}
\setlength {\textwidth} {7.5in} 
\setlength {\textheight}{10.0in}
\setlength{\oddsidemargin} {-0.5in}
\setlength {\evensidemargin} {-0.5in}
\setlength{\topmargin} {-0.0in} 
\setlength {\headheight} {4.0pt} 
\setlength {\headsep} {0in}
\setlength{\columnsep}{0.5cm}


\begin{document}


\begin{sagesilent}
import random
# Generate and set a random seed.
t=ZZ.random_element(999999)
set_random_seed(t)
# Generate lists of random integers to be used
a=[(-1)^(ZZ.random_element(1,3))*ZZ.random_element(2,10) for i in range(20)]
b=[ZZ.random_element(2,9) for i in range(len(a))]
c=[ZZ.random_element(2,8) for i in range(len(a))]
d=[(-1)^(ZZ.random_element(1,3))*ZZ.random_element(2,11) for i in range(len(a))]
f=[ZZ.random_element(1,8) for i in range(len(a))]
g=[ZZ.random_element(2,5) for i in range(len(a))]
h=[ZZ.random_element(2,6) for i in range(len(a))]
k=[(-1)^(ZZ.random_element(1,3))*ZZ.random_element(1,12) for i in range(len(a))]
m=[ZZ.random_element(2,4) for i in range(len(a))]
n=[ZZ.random_element(2,5) for i in range(len(a))]
q=[ZZ.random_element(1,10) for i in range(len(a))]
r=[(-1)^(ZZ.random_element(1,3))*ZZ.random_element(1,10) for i in range(len(a))]

#generate random +/- for LateX output
PM=('+','-')
RPM=[ZZ.random_element(0,2) for i in range(len(a))]
RPM2=[ZZ.random_element(0,2) for i in range(len(a))]
RPM3=[ZZ.random_element(0,2) for i in range(len(a))]
RPM4=[ZZ.random_element(0,2) for i in range(len(a))]


# generate the solution of each equation

# Problem Type1: ab^m + cd^n 
answer1=[a[i]*b[i]^m[i]+(-1)^RPM[i]*c[i]*(d[i])^n[i] for i in range(len(a))]

# Problem Type2: a^m + b \div c^n + f     LT2 is of the form b=d*c^n so that b/c^n=d for nice answer   
answer2=[d[i]^n[i]+(-1)^RPM[i]*g[i]+(-1)^RPM2[i]*c[i]  for i in range(len(a))]
LT2=[g[i]*h[i]^m[i] for i in range(len(a))]

# Problem Type3: a+b[c(d+f)+g(h+k)]
answer3=[q[i] + (-1)^RPM2[i]*h[i]*(m[i]*(a[i] + (-1)^RPM4[i]*c[i]  ) +(-1)^RPM3[i]* n[i]*( d[i] + (-1)^RPM[i]*b[i]    )   )   for i in range(len(a))]

# Problem Type4: (ab+cd)/(f-g)
# Find result of numerator first
# Check if numerator has several divisors and then pick from somewhere in middle of list of divisors
# Create numbers for bottom that ensures denominators results in selected divisor
num4=[  (f[i]*a[i]+ (-1)^RPM[i] *q[i]*d[i] ) for i in range(len(a))]
#print(num4)
den4=[1]*len(a)
for i in range(len(a)):
  if number_of_divisors(num4[i]) > 3:
    den4[i] = divisors(num4[i])[number_of_divisors(num4[i])-3]
  else:
    den4[i] = 1

#print(den4)
den4T=[  (den4[i]-(-1)^RPM2[i]*h[i] )  for i in range(len(a))]
answer4=[  num4[i]/den4[i]   for i in range(len(a))]



# assign variables to a null character
qoutput1=""
aoutput1=""
qoutput2=""
aoutput2=""
qoutput3=""
aoutput3=""
qoutput4=""
aoutput4=""


# generate the latex code for the questions and answers
for i in range(len(a)):
  qoutput1 += r" \item $%s \cdot (%s)^{%s} %s %s \cdot (%s)^{%s}$" %(latex(a[i]),latex(b[i]),latex(m[i]),latex(PM[RPM[i]]),latex(c[i]),latex(d[i]),latex(n[i]))
  aoutput1 += r" \item $%s$" %(latex(answer1[i]))
  qoutput2 += r" \item $(%s)^{%s} %s %s \div %s^{%s} %s %s$"   %(latex(d[i]),latex(n[i]),latex(PM[RPM[i]]),latex(LT2[i]),latex(h[i]),latex(m[i]),latex(PM[RPM2[i]]),latex(c[i]))
  aoutput2 += r" \item $%s$" %(latex(answer2[i]))
  qoutput3 += r" \item $ %s %s %s [%s (%s %s %s) %s %s ( %s %s %s )   ] $"   %(latex(q[i]),latex(PM[RPM2[i]]) , latex(h[i]), latex(m[i]),latex(a[i]) , latex(PM[RPM4[i]]),latex(c[i]),latex(PM[RPM3[i]]),latex(n[i]), latex(d[i]),latex(PM[RPM[i]]) , latex(b[i]) )
  aoutput3 += r" \item $%s$" %(latex(answer3[i]))
  qoutput4 += r" \item $ \frac{%s (%s) %s %s ( %s)}{%s %s %s} $"   %(latex(f[i]),latex(a[i]),latex(PM[RPM[i]]),latex(q[i]),latex(d[i]),latex(den4T[i]),latex(PM[RPM2[i]]),latex(h[i]) )
  aoutput4 += r" \item $%s$" %(latex(answer4[i]))
  
  
\end{sagesilent}



%%%%%%%%%%%%%%%%%%%%  Latex Output %%%%%%%%%%%%%%%%%%%%

% Random Seed can be used to re-create a particular worksheet
\noindent Use the order of operations to simplfy each expression: \\
\textcolor{gray}{Random Seed: $\sage{t}$}


% Create list of problems via qoutput
\begin{multicols}{3}
\begin{enumerate}
\sagestr{qoutput1}
\sagestr{qoutput2}
\sagestr{qoutput3}
\sagestr{qoutput4}
\end{enumerate}
\end{multicols}


\vfill
\newpage


% Create associated list of solutions via aoutput
\begin{multicols}{3}
\begin{enumerate}
\sagestr{aoutput1}
\sagestr{aoutput2}
\sagestr{aoutput3}
\sagestr{aoutput4}
\end{enumerate}
\end{multicols}


\end{document}